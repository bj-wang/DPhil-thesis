%% Section 0: Declaration
%% 第〇节:声明
% 
% LaTeX template of the DPhil thesis, Bingjun Wang, Department of Physics, University of Oxford, May 2021
% 博士论文 LaTeX 模板,王炳俊,牛津大学物理系,2021 年 5 月
% 
% Main references: (1) John McManigle. ociamthesis - Oxford thesis template. https://www.oxfordechoes.com/oxford-thesis-template/.
% Main references: (2) Wei Hu. A Comprehensive Handbook of LaTeX2e (2nd edn., in Chinese). Tsinghua University Press, 2013.
% Main references: (3) Wei Hu. A Handbook of Classes and Packages in LaTeX2e (in Chinese). Tsinghua University Press, 2017.
% 主要参考文献:(1)John McManigle. ociamthesis - Oxford thesis template. https://www.oxfordechoes.com/oxford-thesis-template/.
% 主要参考文献:(2)胡伟. LaTeX2e 完全学习手册(第二版). 清华大学出版社, 2013.
% 主要参考文献:(3)胡伟. LaTeX2e 文类和宏包学习手册. 清华大学出版社, 2017.
% 
% This file is licensed under a Creative Commons Attribution-NonCommercial-NoDerivatives 4.0 International License: https://creativecommons.org/licenses/by-nc-nd/4.0/
% 本文件受到署名-非商业性使用-禁止演绎 4.0 国际知识共享协议的保护:https://creativecommons.org/licenses/by-nc-nd/4.0/deed.zh
% 
% Please contact the author (bingjun.wang@physics.ox.ac.uk or bingjun.wang1995@outlook.com) if you have any questions or comments.
% 如有问题或意见,请联系作者:bingjun.wang@physics.ox.ac.uk 或 bingjun.wang1995@outlook.com
% 
% 
%% Section 1: Preamble
%% 第一节:导言
% 
% Select the document class
% 选择文类
\documentclass{BW-DPhil-thesis-class}
% Input necessary information
% 输入必要的信息
\title{Towards High-Performance Organic Semiconductor Devices: Microstructure, Interlayers, and Emerging Applications}
\author{Bingjun Wang}
\college{New College}
\termyear{Trinity 2021}
\Oxfordlogopath{figures/Oxford-brand.png}
\addbibresource{BW-DPhil-thesis-ref.bib}
% 
% 
%% Section 2: Document Text
%% 第二节:正文
% 
\begin{document}
% Section 2.1: Preliminary matter    % 第 2.1 节:前文区
\maketitleBW    % generate the cover page    % 生成首页
\pagenumbering{roman}    % set page number format    % 设置页码格式
\include{text/abstract}    % generate the abstract    % 生成摘要
\include{text/acknowledgement}    % generate the acknowledgement    % 生成致谢
% generate the table of contents    % 生成目录
{
	\cleardoublepage
	\pdfbookmark[0]{\contentsname}{toc}
	\hypersetup{linkcolor=black}
	\tableofcontents
}
% generate the list of figures and tables    % 生成图表目录
{
	\hypersetup{linkcolor=black}
	\phantomsection
	\begin{spacing}{1.5}
		\listoffigures
	\end{spacing}
	\addcontentsline{toc}{chapter}{List of Figures and Tables}
}
\include{text/abbreviations-and-symbols}    % generate the list of abbreviations and symbols    % 生成缩写和符号表
%
% Section 2.2: Text    % 第 2.2 节:正文文本
\pagenumbering{arabic}    % set page number format    % 设置页码格式
\include{text/Chapter-1-Introduction}
\include{text/Chapter-2-Background-Theory}
\include{text/Chapter-3-Experimental-Methods}
\include{text/Chapter-4-Copolymer}
\include{text/Chapter-5-CuSCN}
\include{text/Chapter-6-HF-Diodes}
\include{text/Chapter-7-Conclusions-and-Outlook}
% 
% Section 2.3: Endmatter (bibliography)    % 第 2.3 节:后文区(参考文献)
\phantomsection
\begin{spacing}{1.25}
	\printbibliography
\end{spacing}
\addcontentsline{toc}{chapter}{Bibliography}
% 
\end{document}